\openingarticle
\def\ppages{\pagerange{bekteshi:firstpage}{bekteshi:lastpage}}
\def\shorttitle{Albanian Archaeology during Communism}
\def\maintitle{Albanian Archaeology during Communism: Constructing the Illyrian Myth through Numismatics}
\def\shortauthor{Arba Bekteshi}
\def\authormail{arba.bekteshi@gmail.com}
\def\affiliation{University of Tirana}
\def\thanknote{\footnote{Arba Bekteshi is currently completing an MSc in Archaeology at the University of Tirana. She has a double undergraduate degree in Southeastern European Studies and International Relations from the American University in Bulgaria, an M. A. in Anthropology of Development and Social Transformation from the University of Sussex, a Master in International Communication from IULM University. At present, she is working on the way post-communist Albanian interpretations are informed by the ideological inheritance of a Marxist-Leninist secular religion and a utopic overreliance on ancient syncretism on Illyrian mythology, reflecting efforts for an apolitical respiritualizing identifier.}}
%--------------------------------------------------------------
\mychapter{Albanian Archaeology during Communism:\newline Constructing the Illyrian Myth through Numismatics}
\begin{center}
	{\Large\scshape\shortauthor \thanknote}\\[1em]
	\email \\
	\affiliation
\end{center}
\vspace{3em}
\midarticle
%--------------------------------------------------------------
\label{bekteshi:firstpage}
%

	%----------------------------------------------------------------------------------------
	\begin{myabstract}  
		In this article,\marginnote{Abstract\\(in Albanian see below)} I argue that Albanian archaeology is in need of self-reflexivity to better interact with its findings and conclusions during the country's communist past, placing itself within the wider international map of theoretically-informed archaeological practice. Drawing on national and international critiques by Richard Hodges, Mark Petruso, Sally Martin, and others, this paper aims to deconstruct the ideological discourses behind interpretations on Illyrian numismatics in the territory of Albania, while assessing the neo-colonialist rhetoric. Past efforts of Albanian archaeologists to construct a politically dictated historicity are placed in the wider context of highly instrumentalized nationalist interpretations of the archaeological record in Europe. 
		
\keywords[Keywords]{Illyrian numismatics, communism, Albanian archaeology, self-reflexivity, nationalism.}
	\end{myabstract}
	


	
\lettrine[nindent=0em,lines=3]{I}{nterpretative}  problems deriving from the historical development of Albanian archaeology can be compared to, and ultimately placed in the map of wider developments in \nth{19} century Europe. Consequently, interpretative constructs in Albanian nationalist archaeological practice are considered an analytical subject in the wake of a European archaeology. Universal traits of Albanian archaeology are drawn in relation to the theoretical impediments it inherits today. I focus on interpretations of Illyrian numismatics, and provide deconstructive ground to such analysis, not to applaud a politically correct archaeology informed by a genuine postmodern perspective, but to offer a possibility for closer readings of past interpretations of Illyrian numismatics. 
	
		
%	\section{The development of Albanian archaeology}
	
In this\marginnote{The development of Albanian archaeology} section I provide an overview of the development of archaeology in Albania and the main factors that influenced it. Apart from the political engagement that archaeology assumed since its early beginnings, I take a look at how the discipline came to bear characteristics specific to its social dimension, regional traditions and, more precisely, to the particularities of its practice in nation-states. While arguing that nationalism remains a contemporary characteristic of southeast Europe \parencite{Korkuti1993}, I refer here mainly to the limitations of archaeology during communism.
	Following sporadic explorations during the Napoleonic wars, the first instances of archaeological practice in Albania were a result of rivalry among the Great Powers to establish their satellites in the newly-freed Balkans of the \nth{19} century. The Austro-Hungarian Empire conducted ``the earliest study of the land, its people and linguistics, with an emphasis on the possibilities of Illyrian survival in the actual Albanian population. These initiatives were part of a wider 'Illyrian' phenomenon linked to the emerging national consciousness of the [\ldots] cosmopolitan Austro-Hungarian world'' \parencite[40]{Gilkes2006}. France and Italy followed with new archaeological missions to trace the Trojans of Epirus. An awareness of archaeology as a means for historical self-determination began to transpose in the Albanian context together with a perceived threat of foreign imposition on its territories, especially after the Conference of London in 1913 demarcated the country's borders.
After the establishment of a communist regime in Albania in 1946, issues such as national pride, sovereignty and political ideology, as declared by the ruling party, were at the centre of scholarly studies. Simultaneously, reproductions of Marxist discourses were a laissez-passer for those wishing to carry out scientific activities. The regime needed to nurse a narrative on a by-gone greatness that would simultaneously feed the propagandistic fears of capitalist invasion and inspire industrial progress, following the soviet ethnogenetic model of the 1930s \parencite[149]{Hodges2004}. Such was the influence of the communist state on Albanian archaeology that today the latter is considered a perfect research case on historical-materialist stripe \parencites[8]{Galaty2006}[703]{Korkuti1993}.
	
	The main objectives of the dictatorship were the assignment of geopolitical boundaries, the creation of a nation state, and the construction of a cultural identity. According to \textcite[323]{Veseli2006}, the geopolitical location of Albania within the Balkans helps explain the need to, and the importance of, constructing a national identity. Moreover, the process of nation-state formation strongly influenced the institutionalisation of Albanian archaeology. Thirdly, archaeological heritage itself has arguably been paramount in the construction and reinforcement of a national and cultural identity \parencite{Veseli2006}. The coexistence of the discipline with the dictatorship created logical vacuums while following a precise agenda to reach preordained conclusions on Illyrian artefacts. Moreover, historical material not belonging to a specific country was vested with expedient meaning \parencite[225]{Kohl1998}.
	
	
%	\section{Interpretations of numismatics in Illyrian territories}
	
	In this\marginnote{Interpretations of numismatics in Illyrian territories} section, I argue that Albanian archaeologists during communism managed to construct an interpretative and material superstructure concerning the Illyrians. In my research on institutional publications from Albanian and international archaeologists, I have encountered three main suppositions in numismatic literature concerning Illyrian numismatic chronology, trade activities, and autochthony. The history of the development of numismatics was transformed in a self-affirmation for both an isolated country and discipline. Ethnocentric interpretations on numismatics claiming that Illyrians grew toward a more complicated culture were the norm. Before dwelling further on the specific case study, it must be stated that I have taken upon interpretations of Illyrian numismatics as a single subject to reflect upon the single grand narrative portrayed during communism.
	
%	\subsection{Chronology}
	
	As far as\marginnote{Chronology} absolute dating of artefacts in Albania is considered, the first examinations were undertaken only recently. The chronology used by Albanian archaeologists is floating and periodicities were devised within the country \parencite[271]{Aliu1985}. Printed symbols and monograms were used to place Illyrian numismatics in analogue categories \parencite[369]{Ceka1974}. To this day, these categories are the best reference in the country \parencite[242]{Sasianu1987}. Inconsistencies arise from literature, to state that numismatic material from the fortress of Mavrove is used to ``establish a compact chronology [\ldots] of buildings and in general of the main phases of inhabitancy \parencite[trans.][65]{Dautaj1981}.'' The treasure at Jubica is used to establish the chronological order of Illyrian drachms \parencite{Ceka1971}; the workshop in Dyrrah is still an unresolved chronological issue \parencite{Ceka1974}; and the problem of chronological arranging of Illyrian cities remains unresolved \parencites{Islami1972}{Islami2008}.
	
%	\subsection{Trade}

Plenty of\marginnote{Trade} other numismatic interpretations from relevant archaeologists argue that Illyrians were active traders \parencites{Mano1986}{Mano2006}{Ceka1965}{Ceka1974}{Gjongecaj1985}{Gjongecaj1986}{Gjongecaj1990}{Picard1986}{Picard1995}. Specifically, \textcite[94]{Prendi1985} argues for the possibility of trade being the decisive factor in the appearance of social stratification during the late Bronze Age, around XV--XIII\BC. During the V--IV\ centuries \BC, the presence of numismatic material is attributed to the trade of a series of cities such as Thronian, Oidantion, Pelion and other unidentified ones with Apollonia, Dyrrah and Epirus \parencite[112--113]{Prendi1974}. \textcite{Mano1974} argues that different cities were specialized in different activities. Not only do ``facts demonstrate coordination and collaboration in monetary activities [\ldots], but also a coordination of trade activities that cities undertook in markets and respective areas'' \parencite[trans.][388]{Mano1974}. The powerful voices of Albanian archaeology, would state that ``the progressive Illyrianisation process of [Dyrrah and Apollonia] in the III--II\ centuries \BC, as well as the representation of an ever increasing number of Illyrian names on [their] coins, gives us the right to talk about this coinage [\ldots] as an Illyrian-Greek one'' \parencite[trans.][164]{Stripceviq1980}.

%	\subsection{Autochthony}
	
	The\marginnote{Autochthony} concept of Albanian autochthony stretched across interpretations to include a mythical materialization of Illyrian presence with special relation to their neighbours. Thus, \textcite{Tzouvara-Souli1993} used representations on coins found in Caonia to state that Epirus and Albania shared common cults. According to \textcite[342]{Waibank1974}, during the III--II\ century BC, Illyrians were bilingual, as demonstrated by their numismatic material. Further interpretations stated that shared Illyrian-Greek cults were not to be excluded \parencite[147]{Meta2006}. To reinforce statements on autochthony, the vocabulary used in Albanian academic research during communism placed Illyrians in a race of cultural evolution, according to which the latter were crystallized groups \parencite[95]{Prendi1985}, with “new concepts on the spiritual world” \parencite[trans.][271]{Aliu1985}.
	
	Contrariwise, ancient authors stated that Illyrians often stole, and were not easily influenced by neighbouring cultures \parencites{Islami2002}{Ceka1974}{Anamali1987}. They also reported that Illyrians were never at the centre of main historical events and very little was written about them \parencite[371]{Franke1974}. Due to antagonistic Illyrian behaviour ``since the middle of the II century \BC, and especially during the 1st century BC, Romans rendered life unbearable in Illyrian territories, which resulted in a series of revolutions'' \parencite[trans.][6]{Anamali1987}. 
		
%	\section{Need for self-reflexivity}
	
	In this\marginnote{Need for self-reflexivity} section, I demonstrate how arguments for a self-reflexive discipline informed by European archaeology bear the same applicability in the Albanian case, paradoxically placing the latter in the wider map of European archaeology. I also argue that after the collapse of the communist regime in Albania in the beginnings of the 1990s, archaeologists felt the need to vindicate their discipline and render justice to archaeological troves.
	In the wake of democracy waves through southeast Europe, Albanian archaeologists were quick to denounce the abuse during communism. ``Many examples could be cited of professional backwardness in Albanian archaeology, of sterile polemic, and of unwillingness to make use of archaeological data from other parts of the world'' \parencite{Miraj1993}. Many were the professionals that understood that self-reflexivity is not a specificity of Albanian archaeology, but that archaeologists have the obligation to reflect on the nature of their subject \parencite[60]{Gramsch2011}.
	Reflecting on teaching processes that provide space for critical engagement in archaeology, \textcite[287]{Hamilakis2004} argues that one of the possible ways to rethink the current instrumentalist pedagogy in archaeology is through the creation of spaces for critical reflection, which link experience and knowledge with subjectivity. These spaces would allow students not only to understand the social and material processes of interpretive subjectivity, but also to challenge and transform these processes and their conditions.
	Along these lines, \textcite[371]{Martin2006} argues that a radical revaluation of Albanian archaeology has been considerably restricted due to significant limitations in the available archaeological research base and the research undertaken, which has fostered the perpetuation of politicised site interpretations, no longer of relevance. Professionals as well as their students still refer to publications made during communism \parencite{Hodges2004}. The latter texts still provide the authority needed to pose a case study \parencite[8–9]{Bintliff2011}, although largely untested and perhaps incorrect \parencite[11]{Galaty2006}. Additionally, it is possible to argue that archaeology in Albania faces slight forms of positivism, given that most of the policymaking, academic and archaeology related institutions have not gone through important phases of democratization. As \textcite[45]{Johnson2010} remarks, single-method positivism masks `institutional intimidation' as neutral praxis, hindering the development of science, and encourages scientism and the `cult of the expert'.
	A new awareness from Albanian archaeologists after the fall of communism is increasingly present, and international collaborations are widely favoured nowadays, but it does not suffice to reach better scientific standards and to radically transform the current academic landscape.
				
%	\section{Conclusion}
	
	Although\marginnote{Conclusion} the need for a more self-reflexive discipline would largely benefit archaeology in Albania, it is imperative not to transpose a series of Western stereotypes informed by the need to replace the Stalinist model \parencite[161]{Hodges2004}. In a context of power imbalances (financial, technological, ideological), representatives of `imperialist' national archaeologies \parencite[Trigger 1984 cited in][]{Galaty2006} may use their critique of archaeological research conducted under dictatorship as support for their own interpretations \parencite[13]{Galaty2006}. This is particularly dangerous given the relativist approach to the past proclaimed by some post-processual archaeologists, by which all interpretations of the archaeological record might be said to be equally valid, regardless of whether they were produced during a democracy or under a dictatorship \parencite[Meskell 1998 cited in][]{Galaty2006}. These post-modern discourses \parencite[Meskell 1998 cited in][]{Galaty2006} have to take into account that the very materiality of the archaeological record constraints the range of interpretations that can be inferred from it \parencite[200]{Hodder1999}. 
	As Kohl and Fawcet 1995b  \parencite[cited in][]{Galaty2006} have argued, we must, as professional archaeologists, face these ethical issues with some sense of responsibility towards the discipline.
	
	In this paper I focused on the control that Albanian archaeology experienced during the communist regime, while serving as a propagandistic tool. I also delved on the importance of self-reflexivity in Albanian archaeology, as a need that has presented worldwide making for a universal trait of the discipline. To conclude, I argued that attempting to construct the Illyrian myth through the use of numismatic material has led to equalizing the Illyrian archaeological culture to the Albanian human one, and rendering archaeological research a cultural historical approach \parencite[18]{Johnson2010}. The challenging deconstruction of these political narratives offers an interesting challenge for the future of the practice of archaeological research in Albania.
			
\myseparator
\begin{myabstract}  
		\foreignlanguage{albanian}{
			Në këtë \marginnote{Abstract (Albanian)}artikull pohoj se arkeologjia shqiptare ka nevojë për vetë-reflektueshmëri për të ndërve\-pruar më mirë me gjetjet dhe përfundimet e saj gjatë të kaluarës komuniste të vendit, duke u vetëvendosur në hartën më të gjerë ndërkombëtare të praktikës arkeologjike të informuar nga teoria. Duke u bazuar te kritikat kombëtare dhe ndërkombëtare nga Richard Hodges, Mark Petruso, Sally Martin dhe të tjerë, ky artikull përpiqet të dekonstruktojë diskurset ideologjike prapa interpretimeve mbi numizmatikën ilire në territorin e Shqipërisë, duke rrahur retorikën neokolonialiste. Përpjekjet e kaluara të arkeologëve shqiptarë për të ngritur një historicitet të diktuar politikisht vendosen në kontekstin më të gjerë të interpretimeve nacionaliste tejet të instrumentalizuara mbi të dhënat arkeologjike në Evropë.}
			
			\keywords[Fjalë kyçe]{numizmatika ilire, komunizmi, arkeologjia shqiptare, vetë-reflektueshmëri.}
				
	\end{myabstract}

	%	REFERENCE LIST

\printbibliography[heading=subbibnumbered] 
\label{bekteshi:lastpage}
\closingarticle
